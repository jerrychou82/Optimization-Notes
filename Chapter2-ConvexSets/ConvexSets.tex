\def\draft{0}
\documentclass[11pt]{article}
\usepackage{amsfonts, fullpage, rotating, amssymb}
\usepackage{amsmath,enumitem}
\usepackage{hyperref}
\usepackage[nameinlink]{cleveref}
\usepackage{tcolorbox}
\tcbuselibrary{xparse}
\newtheorem{theorem}{Theorem}
\newtheorem{conjecture}[theorem]{Conjecture}
\newtheorem{definition}[theorem]{Definition}
\newtheorem{lemma}[theorem]{Lemma}
\newtheorem{proposition}[theorem]{Proposition}
\newtheorem{corollary}[theorem]{Corollary}
\newtheorem{claim}[theorem]{Claim}
\newtheorem{fact}[theorem]{Fact}
\newtheorem{openprob}[theorem]{Open Problem}
\newtheorem{remk}[theorem]{Remark}
\newtheorem{exmp}[theorem]{Example}
\newtheorem{apdxlemma}{Lemma}

\newenvironment{example}{\begin{exmp}
\begin{normalfont}}{\end{normalfont}
\end{exmp}}

\newenvironment{remark}{\begin{remk}
\begin{normalfont}}{\end{normalfont}
\end{remk}}
\newtheorem{sublemma}[theorem]{Sublemma}

\newenvironment{intuition}[1][]{\begin{tcolorbox}[title=Intuition (#1),fonttitle=\bfseries]}{\end{tcolorbox}}
\DeclareTColorBox{conclusion}{}{fonttitle=\bfseries\sffamily\large, title=Conclusion, colframe=red!75!black,colback=red!5!white}

%%%%%%%%%%%%%%%%%%%% proof environments

\def\FullBox{\hbox{\vrule width 8pt height 8pt depth 0pt}}

\def\qed{\ifmmode\qquad\FullBox\else{\unskip\nobreak\hfil
\penalty50\hskip1em\null\nobreak\hfil\FullBox
\parfillskip=0pt\finalhyphendemerits=0\endgraf}\fi}

\def\qedsketch{\ifmmode\Box\else{\unskip\nobreak\hfil
\penalty50\hskip1em\null\nobreak\hfil$\Box$
\parfillskip=0pt\finalhyphendemerits=0\endgraf}\fi}

\newenvironment{proof}{\begin{trivlist} \item {\bf Proof:~~}}
  {\qed\end{trivlist}}

\newenvironment{proofsketch}{\begin{trivlist} \item {\bf
Proof Sketch:~~}}
  {\qedsketch\end{trivlist}}

\newenvironment{proofof}[1]{\begin{trivlist} \item {\bf Proof
#1:~~}}
  {\qed\end{trivlist}}

\newenvironment{claimproof}{\begin{quotation} \noindent
{\bf Proof of claim:~~}}{\qedsketch\end{quotation}}


%%%%%%%%%%%%%%%%%%%%%%% text macros
\newcommand{\etal}{{\it et~al.\ }}
\newcommand{\ie} {{\it i.e.,\ }}
\newcommand{\eg} {{\it e.g.,\ }}
\newcommand{\cf}{{\it cf.,\ }}

%%%%%%%%%%%%%%%%%%%%%%% general useful macros
\newcommand{\eqdef}{\mathbin{\stackrel{\rm def}{=}}}
\newcommand{\R}{{\mathbb R}} % real numbers
\newcommand{\N}{{\mathbb{N}}} % natural numbers
\newcommand{\Z}{{\mathbb Z}} % integers
\newcommand{\F}{{\mathbb F}} % a field
\newcommand{\Q}{{\mathbb Q}} % the rationals
\newcommand{\poly}{{\mathrm{poly}}}
\newcommand{\polylog}{{\mathrm{polylog}}}
\newcommand{\loglog}{{\mathop{\mathrm{loglog}}}}
\newcommand{\zo}{\{0,1\}}
\newcommand{\suchthat}{{\;\; : \;\;}}
\newcommand{\pr}[1]{\Pr\left[#1\right]}
\newcommand{\deffont}{\em}
\newcommand{\getsr}{\mathbin{\stackrel{\mbox{\tiny R}}{\gets}}}
\newcommand{\Exp}{\mathop{\mathrm E}\displaylimits} % expectation
\newcommand{\Var}{\mathop{\mathrm Var}\displaylimits} % variance
\newcommand{\xor}{\oplus}
\newcommand{\GF}{\mathrm{GF}}
\newcommand{\eps}{\varepsilon}


\pagestyle{plain}

\newcommand{\scribe}{Chi-Hua Wang, Chi-Ning Chou}
\newcommand{\lecnum}{2}
\newcommand{\lecname}{Convex Sets}
\newcommand{\lecdate}{\today}

%\parskip=1.5mm
%\parindent=0mm

\begin{document}
	
\begin{center}
	\renewcommand{\arraystretch}{2}
	\begin{bfseries}
		\begin{tabular}{c}
			\vspace{0.5cm}
			\Huge Convex Optimization\\
			\vspace{0.5cm}
			\hspace{10em} {\Large Chapter \lecnum: \lecname} \hspace{10em}\ \\
			\lecdate \hfill \scribe\\
			\hline
		\end{tabular}
		\renewcommand{\arraystretch}{1}
	\end{bfseries}
\end{center}

\section{Convex Sets}

\subsection{Definition and convexity}


\subsection{Examples}

\paragraph{2.14} ({\it expanded and restricted sets})\\
\concept Scaling a set (either extending or shrinking) preserves convexity.\\
\proofidea Drawing a trapezoid. 


\subsection{Operations that preserve convexity}
\paragraph{2.16} ({\it partial sum})\\
\concept As sum preserves convexity, partial sum is just summing over a subspace and clearly preserves convexity.

\paragraph{2.18} ({\it Invertible linear fractional function})\\
\concept Consider the {\it Projective interpolation} in page 41.
\details We associate a point in $x\in\R^n$ with a ray $\mathcal{P}(x) = \{t(x,1)|t\geq0\}$. Then we have
\begin{itemize}
	\item $f(x) = \mathcal{P}(Q(\mathcal{P}^{-1}(x)))$
	\item $f^{-1}(y) = \mathcal{P}^{-1}(Q^{-1}(\mathcal{P}))$
\end{itemize}

\subsection{Separation theorem and supporting hyperplanes}
In this subsection we investigate the powerful {\it separating hyperplane theorem}. The theorem states that
$$\mbox{If C and D are {\bf disjoint} convex sets}\Rightarrow \exists\mbox{ separating hyperplane}$$
This can be easily proved by finding one. But this is not the end of the story, we also care about the {\it converse} of the theorem. That is,
$$\mbox{C and D are convex and there's a separting affine function}\Rightarrow \mbox{C and D are disjoint}$$
Another important aspect is to find {\bf alternative} conditions for inequalities.\\
For example: ({\it Alternatives for strict inequalities})
$$Ax\prec b\Leftrightarrow \exists\lambda\mbox{ s.t. }\lambda\neq0,\ \lambda\succeq0,\ A^T\lambda=0,\ \lambda^Tb\leq0$$


\paragraph{2.20} ({\it strictly positive solution of linear equations})\label{2.20}\\
\goal $\exists x\succ0,\ Ax=b\Leftrightarrow\mbox{there's {\bf no }}\lambda\ \mbox{such that } A^T\lambda\succeq0,A^T\lambda\neq0,b^T\lambda\leq0$\\
\details \hyperref[2.20details]{link}
\begin{intuition}[strictly positive solution of linear equations]
	"$\mbox{there's {\bf no }}\lambda\ \mbox{such that } A^T\lambda\succeq0,A^T\lambda\neq0,b^T\lambda\leq0$" is a sufficient condition for the existence of strictly positive solution.
\end{intuition}

\paragraph{2.22}({\it general-case separating hyperplane theorem})\\
\goal If convex sets $C$ and $D$ are disjoint $\Rightarrow$ $\exists$ separating hyperplane.

\paragraph{2.24}({\it examples of supporting hyperplane})\\
\begin{enumerate}[label=(\alph*)]
	\item A convex set $C$ = $\bigcap$ halfspaces constructed by the supporting hyperplanes that contain $C$
	\item How to explicitly write down the supporting hyperplane of infinite norm.
\end{enumerate}

\paragraph{2.26}({\it support function})\\
\begin{definition}[support function]
	The support function of $C\subseteq \R^n$ is
	$$S_C(y) = \sup\{y^Tx|x\in C\}$$
\end{definition}
\goal For two sets $C$ and $D$, $C=D\Leftrightarrow S_C=S_D$.\\
\proofidea If $\exists x\in D,\ x\notin C$, then by {\it separating hyperplane theorem}, there is a hyper-plane $a^Tx+b$ that separates $x$ and $C$. Then either $a^Tx > S_C(a)$ or $-a^Tx > S_C(-a)$.
\begin{intuition}[supporting function]
	The maximizer of the supporting function is the point that being supported by the hyperplane with normal vector $y$.
\end{intuition}


\subsection{Convex cones and generalized inequalities}
Here, we want to use a {\it proper cone} to define a generalized inequality and explore some useful results. First, we define the proper cone 
\begin{definition}[proper cone]
	We say a cone $K\subseteq\R^n$ is proper if
	\begin{itemize}
		\item $K$ is convex. (contains every line segments)
		\item $K$ is closed. (consists every limit points)
		\item $K$ is solid. (has interior point)
		\item $K$ is pointed. (contains no line or contain $\mathbf{0}$)
	\end{itemize}
\end{definition}
Then, we define a partial ordering on $\R^n$ with $K$ as follow
$$x\preceq_Ky\Leftrightarrow y-x\in K$$
, which is called {\it generalized inequality}.
\begin{intuition}[generalized inequality]
	To check whether $x\preceq_Ky$, we can move the center of the cone $K$ to $x$ and see whether $y$ is inside.
\end{intuition}
Next, we define the concept of dual cone
\begin{definition}[dual cone]
	Let $K$ be a cone, then
	$$K^* = \{y: y^Tx\geq0,\ \forall x\in K\}$$
	is the dual cone of $K$.
\end{definition}
\begin{intuition}[dual cone]
	Geometrically, $y\in K^*\Leftrightarrow-y$ is the normal of a hyperplane that supports $K$ at the {\bf origin}.
\end{intuition}
\paragraph{2.28}({\it positive semidefinite cone})\\
\goal Algebraic intuitions for positive semidefinite cone of dimension $n=1,2,3$.
\paragraph{2.30}({\it properties of generalized inequality})
The following is the properties of generalized inequality $\preceq_K$:
\begin{itemize}
	\item Preserved addition.
	\item Transitive.
	\item Preserved nonnegative scaling.
	\item Reflexive.
	\item Antisymmetric.
	\item Preserved under limits.
\end{itemize}

\paragraph{2.32}({\it dual cone of the image of a linear transformation with nonnegative domain})\\
\goal $K = \{Ax: x\succeq0\} \Rightarrow K^*=\{y:A^Ty\succeq0 \}$

\paragraph{2.34}({\it lexicographic cone and ordering})\\
\begin{definition}[lexicographic cone]
	The lexicographic cone is
	$$K_{lexi} = \{0\}\cup\{x\in\R^n: x_1=\cdots=x_k=0,\ x_{k+1}>0,\ \mbox{for some } k \}$$
	Namely, the first nonzero entry is positive.
\end{definition}
Directly, we can define the lexicographic ordering as $x\leq_{lexi}y\Leftrightarrow y-x\in K_{lexi}$.\\
We have
\begin{itemize}
	\item $K_{lexi}$ is a cone but not proper.
	\item $\leq_{lexi}$ is a linear ordering.
	\item $K^*_{lexi} = \R_+e_1 = \{(t,0,...,0):t\geq0\}$
\end{itemize}
\begin{intuition}[lexicographic cone and ordering]
	Lexicographic ordering is like a {\bf dictionary}!
\end{intuition}

\paragraph{2.36}({\it Euclidean distance matrices})\\
\goal The set of Euclidean distance matrices is a convex cone.
\begin{definition}[Euclidean distance matrix]
	We say $D\in\mathbf{S}^n$ is a Euclidean matrix if $\exists x_1,...,x_n\in\R^n$ such that $D_{ij} = ||x_i-x_j||^2_2$.
\end{definition}
Moreover, there's a necessary and sufficient condition for Euclidean distance matrix:
$$D\in\mathbf{S}^n\mbox{ is a Euclidean distance matrix}\Leftrightarrow D_{ii} = 0,\ x^TDx\leq0\ \forall x\ s.t.\ \mathbf{1}^Tx=0$$
\paragraph{2.38}({\it barrier cone, recession cone, normal cone})


\appendix
\section{Details}

\paragraph{2.20}\label{2.20details}\hyperref[2.20]{link}\\
$(\Rightarrow)$ is trivial. ($\Leftrightarrow$) is shown by separating hyperplane theorem as follow:
\begin{enumerate}
	\item If there's no $x\succ0,\ Ax=b$, it means that $\R_{++}^{n}$ and $\{x: Ax=b\}$ are disjoint.
	\item Thus, we can apply {\it separating hyperplane theorem}: $\exists c\neq0,d$ such that 
	\begin{itemize}
		\item $c^Tx\geq d$ for all $x$ in $\R_{++}^n$
		\item $c^Tx\leq d$ for all $x$ in $\{x:Ax=b\}$
	\end{itemize}
	\item By 2., we have
	\begin{itemize}
		\item $c\geq0$ and $d\leq0$.
		\item As $\{x: Ax=b\}$ is {\bf affine}, a linear function either be a constant or takes all value. And here it can only be the constant case. That is, $\exists d'\leq d$ such that $c^Tx=d',\ \forall x\in\{x:Ax=b\}$.
	\end{itemize}
	\item By lemma
	$$c^Tx=d',\ \forall x\in\{x:Ax=b\}\Leftrightarrow\exists\lambda\mbox{ such that }c=A^T\lambda,d'=b^T\lambda$$
	There is a $\lambda$ such that $A^T\lambda=c\geq0$, $A^T\lambda=c\neq0$, and $c^T\lambda=d'\leq d\leq0$
\end{enumerate}


\newpage
%%% Reference%%%
\bibliographystyle{alpha}
\bibliography{mybib}


\end{document}




